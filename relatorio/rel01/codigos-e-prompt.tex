% Options for packages loaded elsewhere
\PassOptionsToPackage{unicode}{hyperref}
\PassOptionsToPackage{hyphens}{url}
\PassOptionsToPackage{dvipsnames,svgnames,x11names}{xcolor}
%
\documentclass[
  letterpaper,
  DIV=11,
  numbers=noendperiod]{scrartcl}

\usepackage{amsmath,amssymb}
\usepackage{iftex}
\ifPDFTeX
  \usepackage[T1]{fontenc}
  \usepackage[utf8]{inputenc}
  \usepackage{textcomp} % provide euro and other symbols
\else % if luatex or xetex
  \usepackage{unicode-math}
  \defaultfontfeatures{Scale=MatchLowercase}
  \defaultfontfeatures[\rmfamily]{Ligatures=TeX,Scale=1}
\fi
\usepackage{lmodern}
\ifPDFTeX\else  
    % xetex/luatex font selection
\fi
% Use upquote if available, for straight quotes in verbatim environments
\IfFileExists{upquote.sty}{\usepackage{upquote}}{}
\IfFileExists{microtype.sty}{% use microtype if available
  \usepackage[]{microtype}
  \UseMicrotypeSet[protrusion]{basicmath} % disable protrusion for tt fonts
}{}
\makeatletter
\@ifundefined{KOMAClassName}{% if non-KOMA class
  \IfFileExists{parskip.sty}{%
    \usepackage{parskip}
  }{% else
    \setlength{\parindent}{0pt}
    \setlength{\parskip}{6pt plus 2pt minus 1pt}}
}{% if KOMA class
  \KOMAoptions{parskip=half}}
\makeatother
\usepackage{xcolor}
\setlength{\emergencystretch}{3em} % prevent overfull lines
\setcounter{secnumdepth}{-\maxdimen} % remove section numbering
% Make \paragraph and \subparagraph free-standing
\makeatletter
\ifx\paragraph\undefined\else
  \let\oldparagraph\paragraph
  \renewcommand{\paragraph}{
    \@ifstar
      \xxxParagraphStar
      \xxxParagraphNoStar
  }
  \newcommand{\xxxParagraphStar}[1]{\oldparagraph*{#1}\mbox{}}
  \newcommand{\xxxParagraphNoStar}[1]{\oldparagraph{#1}\mbox{}}
\fi
\ifx\subparagraph\undefined\else
  \let\oldsubparagraph\subparagraph
  \renewcommand{\subparagraph}{
    \@ifstar
      \xxxSubParagraphStar
      \xxxSubParagraphNoStar
  }
  \newcommand{\xxxSubParagraphStar}[1]{\oldsubparagraph*{#1}\mbox{}}
  \newcommand{\xxxSubParagraphNoStar}[1]{\oldsubparagraph{#1}\mbox{}}
\fi
\makeatother


\providecommand{\tightlist}{%
  \setlength{\itemsep}{0pt}\setlength{\parskip}{0pt}}\usepackage{longtable,booktabs,array}
\usepackage{calc} % for calculating minipage widths
% Correct order of tables after \paragraph or \subparagraph
\usepackage{etoolbox}
\makeatletter
\patchcmd\longtable{\par}{\if@noskipsec\mbox{}\fi\par}{}{}
\makeatother
% Allow footnotes in longtable head/foot
\IfFileExists{footnotehyper.sty}{\usepackage{footnotehyper}}{\usepackage{footnote}}
\makesavenoteenv{longtable}
\usepackage{graphicx}
\makeatletter
\def\maxwidth{\ifdim\Gin@nat@width>\linewidth\linewidth\else\Gin@nat@width\fi}
\def\maxheight{\ifdim\Gin@nat@height>\textheight\textheight\else\Gin@nat@height\fi}
\makeatother
% Scale images if necessary, so that they will not overflow the page
% margins by default, and it is still possible to overwrite the defaults
% using explicit options in \includegraphics[width, height, ...]{}
\setkeys{Gin}{width=\maxwidth,height=\maxheight,keepaspectratio}
% Set default figure placement to htbp
\makeatletter
\def\fps@figure{htbp}
\makeatother

\KOMAoption{captions}{tableheading}
\makeatletter
\@ifpackageloaded{caption}{}{\usepackage{caption}}
\AtBeginDocument{%
\ifdefined\contentsname
  \renewcommand*\contentsname{Table of contents}
\else
  \newcommand\contentsname{Table of contents}
\fi
\ifdefined\listfigurename
  \renewcommand*\listfigurename{List of Figures}
\else
  \newcommand\listfigurename{List of Figures}
\fi
\ifdefined\listtablename
  \renewcommand*\listtablename{List of Tables}
\else
  \newcommand\listtablename{List of Tables}
\fi
\ifdefined\figurename
  \renewcommand*\figurename{Figure}
\else
  \newcommand\figurename{Figure}
\fi
\ifdefined\tablename
  \renewcommand*\tablename{Table}
\else
  \newcommand\tablename{Table}
\fi
}
\@ifpackageloaded{float}{}{\usepackage{float}}
\floatstyle{ruled}
\@ifundefined{c@chapter}{\newfloat{codelisting}{h}{lop}}{\newfloat{codelisting}{h}{lop}[chapter]}
\floatname{codelisting}{Listing}
\newcommand*\listoflistings{\listof{codelisting}{List of Listings}}
\makeatother
\makeatletter
\makeatother
\makeatletter
\@ifpackageloaded{caption}{}{\usepackage{caption}}
\@ifpackageloaded{subcaption}{}{\usepackage{subcaption}}
\makeatother

\ifLuaTeX
  \usepackage{selnolig}  % disable illegal ligatures
\fi
\usepackage{bookmark}

\IfFileExists{xurl.sty}{\usepackage{xurl}}{} % add URL line breaks if available
\urlstyle{same} % disable monospaced font for URLs
\hypersetup{
  pdftitle={Codigos e Prompt},
  pdfauthor={Nicolas Willian Ribeiro},
  colorlinks=true,
  linkcolor={blue},
  filecolor={Maroon},
  citecolor={Blue},
  urlcolor={Blue},
  pdfcreator={LaTeX via pandoc}}


\title{Codigos e Prompt}
\author{Nicolas Willian Ribeiro}
\date{}

\begin{document}
\maketitle


\subsection{Codigos Utilizados no
Rstudio}\label{codigos-utilizados-no-rstudio}

Os codigos abaixo foram gerados pela Ia ChatGPT e foram editados para
que não serem apresetados diretamente na pagina do relatorio.\\
\strut \\
(\#)\{r setup, include=FALSE\}\\
knitr::opts\_chunk\$set(echo = FALSE)

(\#)Instale o pacote `modeest' se ainda não tiver\\
if(!require(modeest)) install.packages(``modeest'')\\
library(modeest) library(knitr)

(\#)Dados coletados ponto \textless- 1:20 sinal\_5g \textless- c(-65,
-70, -68, -72, -74, -80, -77, -69, -75, -73, -71, -78, -82, -76, -79,
-67, -66, -83, -81, -70)

(\#)Criando tabela tabela\_sinal \textless- data.frame(
\texttt{Ponto\ de\ Coleta} = ponto, \texttt{Intensidade\ (dBm)} =
sinal\_5g )

(\#)Exibindo tabela kable(tabela\_sinal, caption = ``Tabela 1:
Intensidade de Sinal 5G Coletada em Ouro Branco - MG'')

(\#)Estatísticas media \textless- mean(sinal\_5g) mediana \textless-
median(sinal\_5g) moda \textless- mlv(sinal\_5g, method = ``mfv'')
desvio\_padrao \textless- sd(sinal\_5g) variancia \textless-
var(sinal\_5g) minimo \textless- min(sinal\_5g) maximo \textless-
max(sinal\_5g) amplitude \textless- maximo - minimo quartis \textless-
quantile(sinal\_5g) iqr \textless- IQR(sinal\_5g)

\begin{enumerate}
\def\labelenumi{(\arabic{enumi})}
\tightlist
\item
  Tabela resumo tabela\_estatisticas \textless- data.frame( Medida =
  c(``Média'', ``Mediana'', ``Moda'', ``Desvio Padrão'', ``Variância'',
  ``Mínimo'', ``Máximo'', ``Amplitude'', ``1º Quartil (Q1)'', ``3º
  Quartil (Q3)'', ``IQR''), Valor = round(c(media, mediana, moda,
  desvio\_padrao, variancia, minimo, maximo, amplitude, quartis{[}2{]},
  quartis{[}4{]}, iqr), 2) )
\end{enumerate}

(\#)Exibindo tabela kable(tabela\_estatisticas, caption = ``Tabela 2:
Medidas Estatísticas da Intensidade de Sinal 5G'')

hist(sinal\_5g, breaks = 10, col = ``steelblue'', border = ``white'',
main = ``Distribuição da Intensidade de Sinal 5G'', xlab = ``Intensidade
(dBm)'', ylab = ``Frequência'') abline(v = media, col = ``red'', lwd =
2, lty = 2) legend(``topright'', legend = c(``Média''), col = ``red'',
lty = 2, lwd = 2)

boxplot(sinal\_5g, horizontal = TRUE, col = ``lightblue'', main =
``Boxplot da Intensidade de Sinal 5G'', xlab = ``Intensidade (dBm)'')

\subsection{}\label{section}




\end{document}
